%
% This file is part of revolution-installer.
% revolution-installer is free software: you can redistribute it and/or modify
% it under the terms of the GNU General Public License as published by the
% Free Software Foundation, either version 3 of the License, or
% (at your option) any later version.
%
% revolution-installer is distributed in the hope that it will be useful,
% but WITHOUT ANY WARRANTY; without even the implied warranty of
% MERCHANTABILITY or FITNESS FOR A PARTICULAR PURPOSE.
% See the GNU General Public License for more details.
%
% You should have received a copy of the GNU General Public License
% along with revolution-installer. If not, see <https://www.gnu.org/licenses/>.
%
% Copyright 2022 AntaresMKII

\documentclass{scrartcl}

\title{Revolution Installer}
\author{AntaresMkII a.k.a. ComradeKrasMazov}

\begin{document}
\maketitle
\tableofcontents
\section{Introduction}
This document outlines the technical specifications and implementation of the
program revolution-installer developed for Soviet Linux by the Soviet Linux
development team and its front-end, Revolution. If you have inquires on this
document or you wish to suggest changes please do so in our official Discord
server.

\section{Running Revolution-installer}
Revolution-installer comes pre-packaged in the Soviet Linux official ISO image
and can be executed as root user in bash by running the \texttt{revolution}
command. This will execute the Revolution front-end program which will guide
the user trough the installation process using TUI prompts. In addition it will
also provide a "one command" installation process, which consists in executing
the \texttt{revolution} command with a set of arguments to configure the system.

\section{The Revolution-installer API}
The Revolution-installer provides an API to install the system. This API is
specific for the Soviet Linux system, however it could be ported to another
operating system. The API makes use of the CCCP package manager and its library
to install the system and additional packages.

\subsection{Locale}
\subsubsection{setlocale}
The \texttt{setlocale} API call will set the specified locale
\subsubsection{showlocale}
The \texttt{showlocale} API call will display all available locale

\subsection{Keyboard format}
\subsubsection{setkb}
The \texttt{setkb} API call will set the keyboard according to the argument
passed by the user. It follows the standard naming scheme for keyboards
\subsubsection{listkb}
This API call will return a list of keyboard layouts that can be chosen by the
user.

\subsection{Hostname}
To set a user defined hostname use the \texttt{sethost} API call. It will set
the hostname for the system to be installed.

\subsection{Time Zone}
\subsubsection{listzones}
Returns a list of available zones. This API call will work differently depending
on the argument passed. If the argument is an empty string or a NULL pointer
the return will be a list of string representing the continent of the time zone.
Otherwise if the argument is a continent then it will return a list of time
zones available.
\subsubsection{setzone}
Set the time zone. It accepts two strings as arguments: one for the continent
one for the time zone.

\subsection{Root user and User creation}
\subsubsection{setrootpasswd}
This API call sets the root password. It accepts one string argument.
\subsubsection{createuser}
This API call creates a new user. By default revolution-installer will set
the default user settings as defined in \texttt{man useradd}, however it
is possible to add additional arguments to this API call to change its behaviour.
Here is a list of acceptable arguments:
% TODO: Add a list here with all arguments for createuser

This function returns an integer, defined as user id. Store this value as it
will be necessary to complete the user setup.
\subsubsection{setuserpasswd}
This API call sets the user password using the argument passed. It require a
user id to be passed as argument.

\subsection{User Goups}
After a user is created use the following calls to set its groups. It is necessary
for a user to be already created for these calls to work properly.
\subsubsection{listgroups}
Given a user id, returned by \texttt{createuser}, it will return a list of
groups structures. These structures contain the group name as well as a value
indicating if the user is part of these groups: 1 if it is, 0 if it is not.
\subsubsection{setgroups}
Sets the groups for the user. Specify the groups in a list of groups structures
and provide a user id.

\subsection{Network}
As of \date Soviet Linux only officially supports only wired connections, so
revolution-installer will support only those as well. This will be changed in
future versions of the software.
% TODO: Complete network section. When you have internet

\subsection{Installation Medium Selection}
Revolution-installer supports multiple ways to install the system. It is possible
to install Soviet Linux from the local ISO, from the official rootfs tarball or
trough the CCCP. The CCCP option require the system to  be recompiled from scratchbut it will provide the most up to date packages, while the tarball and the ISO
will be faster to install but might be out of date.
\subsubsection{setinstmethod}
This call sets the installation method. It accepts one of the following
arguments:
\begin{description}
    \item[LOCAL] uses the local ISO image to install the system
    \item[TARBALL] uses the rootfs tarball to install the system
    \item[SOURCE] uses CCCP to install the system
\end{description}

\subsection{Disk partition}
Disk partition, in Revolution-installer, occurs trough the use of partx as a
backend. The API will simply act as a buffer between the front-end and partx.
It will also simplify the interface.
\subsubsection{addpart}
Adds a new partition to the specified disk of the defined size. It returns an id
to identify the partition.



\section*{Appendix A: License}
This document ad its source code are distributed under The GNU GPL v3.0. For
more information on this licence visit: https://www.gnu.org/licenses/

Copyright 2022 AntaresMkII

\end{document}
